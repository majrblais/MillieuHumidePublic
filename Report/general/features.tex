\begin{longtable}{|l|p{3cm}|p{10cm}|}
    \hline
    \textbf{Name} & \textbf{Title} & \textbf{Description} \\ \hline
    \endfirsthead
    \multicolumn{3}{c}%
    {{\bfseries \tablename\ \thetable{} -- continued from previous page}} \\
    \hline
    \textbf{Name} & \textbf{Title} & \textbf{Description} \\ \hline
    \endhead
    \hline \multicolumn{3}{r}{{Continued on next page}} \\ \hline
    \endfoot
    \hline
    \endlastfoot
    OF1 & Province & Always New Brunswick in the case of the project. \\ \hline
    OF2 & Accumulated water area within 1 km & Recent orthoimagery (ESRI Hybrid), mosaic of very high spatial resolution aerial photographs, 1 km buffer layer, and area measurement tool. \\ \hline
    OF3 & Accumulated water and wetlands within 1 km & Recent orthoimagery (ESRI Hybrid), mosaic of very high spatial resolution aerial photographs, regulated wetlands layers – 2010 and 2020, 1 km buffer layer, and area measurement tool. \\ \hline
    OF4 & Size of the largest area or corridor of vegetation & Recent orthoimagery (ESRI Hybrid), mosaic of very high spatial resolution aerial photographs, Forest layer (location of plantations), and area measurement tool. Always > 1000 hectares for this project. \\ \hline
    OF5 & Distance from a large expanse of vegetation & Recent orthoimagery (ESRI Hybrid), mosaic of very high spatial resolution aerial photographs, Forest layer (location of plantations), and area and distance measurement tool. 1 – Is there a massif of more than 375 ha of unmanaged vegetation nearby? Yes. 2 – What is the minimum distance between the edge of the evaluation area and the edge of the nearest massif of more than 375 ha of unmanaged vegetation? 3 – Is there a physical separation between the edge of the evaluation area and the edge of the massif of more than 375 ha of unmanaged vegetation? \\ \hline
    OF6 & Uniqueness of herbaceous plants & Recent orthoimagery (ESRI Hybrid), mosaic of very high spatial resolution aerial photographs, and circular buffer layers of 100 m, 1 km, and 5 km radius centered on a point chosen in the evaluation area based on its proximity to a road or main forest path and its accessibility. In our context, the vegetation cover of the evaluation area is usually composed of less than 10\% herbaceous plants (including mosses). In this case, put 0 (P. Adamus, personal communication, 30/06/2020). \\ \hline
    OF7 & Uniqueness of woody cover & Recent orthoimagery (ESRI Hybrid), mosaic of very high spatial resolution aerial photographs, circular buffer layers of 100 m, 1 km, and 5 km radius centered on a point chosen in the evaluation area based on its proximity to a road or main forest path and its accessibility, and Forest layer (location of plantations). In our context, the terrestrial environment within a radius of 5 km, 1 km, and 100 m from the edge of the wetland is composed of more than 10\% woody cover. In this case, put 0 (P. Adamus, personal communication, 04/08/2021). \\ \hline
    OF8 & Percentage of local vegetation cover & Recent orthoimagery (ESRI Hybrid), mosaic of very high spatial resolution aerial photographs, circular buffer layer of 5 km radius centered on a point chosen in the evaluation area based on its proximity to a road or main forest path and its accessibility, and Forest layer (location of plantations). \\ \hline
    OF9 & Type of soil degradation & Recent orthoimagery (ESRI Hybrid), mosaic of very high spatial resolution aerial photographs, circular buffer layer of 5 km radius centered on a point chosen in the evaluation area based on its proximity to a road or main forest path and its accessibility, and Forest layer (location of plantations). \\ \hline
    OF10 & Road distance to the nearest population center & Recent orthoimagery (ESRI Hybrid), mosaic of very high spatial resolution aerial photographs, road network and forest path layers, and linear measurement tool. \\ \hline
    OF11 & Distance to the nearest maintained road & Recent orthoimagery (ESRI Hybrid), mosaic of very high spatial resolution aerial photographs, road network and forest path layers, and linear measurement tool. \\ \hline
    OF12 & Wildlife access & Recent orthoimagery (ESRI Hybrid), mosaic of very high spatial resolution aerial photographs, and circular buffer layer of 5 km radius centered on a point chosen in the evaluation area based on its proximity to a road or main forest path and its accessibility. \\ \hline
    OF13 & Distance from an accumulated water body & Recent orthoimagery (ESRI Hybrid), mosaic of very high spatial resolution aerial photographs, and linear measurement tool. \\ \hline
    OF14 & Distance from a large accumulated water body & Recent orthoimagery (ESRI Hybrid), mosaic of very high spatial resolution aerial photographs, and linear and area measurement tool. Always more than 10 km in the context of the project. \\ \hline
    OF15 & Proximity to tides & Always more than 40 km in the context of the project. \\ \hline
    OF16 & Contact with the terrestrial environment & Recent orthoimagery (ESRI Hybrid), mosaic of very high spatial resolution aerial photographs, and very high spatial resolution digital elevation model. Visual estimation. \\ \hline
    OF17 & Damage caused by non-tidal waters & Use of the layer from the atlas of 2 countries one forest (2C1Forest) called Active river area for the Northern Appalachian-Acadian Region. Add a 5 km buffer around the layer. \\ \hline
    OF18 & Relative elevation in the watershed & Very high spatial resolution digital elevation model. \\ \hline
    OF19 & Watershed sensitive to water quality & Always 1 in the context of this project (designated watersheds). \\ \hline
    OF20 & Degraded water quality upstream & Visual observations, recent orthoimagery (ESRI Hybrid), and mosaic of very high spatial resolution aerial photographs (0.1 m; GéoNB 2018). \\ \hline
    OF21 & Degraded water quality downstream & Visual observations, recent orthoimagery (ESRI Hybrid), and mosaic of very high spatial resolution aerial photographs. \\ \hline
    OF22 & Wetland percentage of the contribution area & New Brunswick wetland layers (2010 and 2020), contribution area layer, and area measurement tool. Reference wetland: local wetland/not the wetland complex. Excel file created for calculations. \\ \hline
    OF23 & Non-vegetated surface in the contribution area & Contribution area layer and area measurement tool. Negligible; significant; important or very important. \\ \hline
    OF24 & Surface from an ascending slope & Recent orthoimagery (ESRI Hybrid), mosaic of very high spatial resolution aerial photographs, hydrographic network layer, very high spatial resolution digital elevation model, and very high spatial resolution slope layer. Soil depth: see Madawaska County soil study (1980). \\ \hline
    OF25 & Orientation & Orientation of the current direction. \\ \hline
    OF26 & Internal flow distance (Flow path length) & Hydrographic network layer + Linear measurement tool to measure the distance between the inlet and outlet of water flowing in the wetland. \\ \hline
    OF27 & Growing degree days & Growing degree days layer. \\ \hline
    OF28 & Fish access or use & No data. \\ \hline
    OF29 & Species conservation concern & See with ACCDC data, eBirds, Ornitho Club, available rare plant data. No data. \\ \hline
    OF30 & Important Bird Area (IBA) & Always 0 in the context of this project (important bird areas layer). \\ \hline
    OF31 & Black duck nesting area & No data. \\ \hline
    OF32 & Concentration areas for wintering deer or moose & On public lands, deer wintering area layer, only for W16 and W17. \\ \hline
    OF33 & Other conservation designation & Natural protected areas layers, provincially significant wetlands, and Nature Conservancy Canada. \\ \hline
    OF34 & Conservation effort & Always 0 in the context of the project. \\ \hline
    OF35 & Mitigation measure & Always 0 in the context of the project. \\ \hline
    OF36 & Sustained scientific use & Always 0 in the context of the project. \\ \hline
    OF37 & Limestone region & Bedrock geology and forest soils layers of NB. \\ \hline
    OF38 & Property type & Crown land layer. Public lands: Activities permitted according to status; 2nd condition = general case. Private lands: 4th condition. \\ \hline
    F1 & Wetland Type & Focus on vegetation type. \\ \hline
    F2 & Adjacent or secondary wetland type & \\ \hline
    F3 & Diversity - Height (vertical stratification of the canopy) and nature of the cover & \\ \hline
    F4 & Dominance of the most abundant shrub species & \\ \hline
    F5 & Diameter classes of woody species & \\ \hline
    F6 & Intermingling of height classes & \\ \hline
    F7 & Large snags (Standing dead trees) & \\ \hline
    F8 & Downed wood & \\ \hline
    F9 & Nitrogen fixer & \\ \hline
    F10 & Extent of Sphagnum Moss & \\ \hline
    F11 & \% of bare soil and thatch/litter & \\ \hline
    F12 & Soil irregularity & Use of the DEM + Shading layer to see topographic variation. \\ \hline
    F13 & Terrestrial inclusion & Area of terrestrial inclusions in the evaluation area. The high-resolution spatial DEM could be used. \\ \hline
    F14 & Soil texture & Use the document “Soils of New Brunswick: The second approximation” (Fahmy et al. 2010) + Soil layer + Excel table to draw a conclusion on soil types and their granulometry. \\ \hline
    F15 & Shorebird feeding habitats & \\ \hline
    F16 & \% Herbaceous part of the wetland vegetation & The mosaic of very high-resolution spatial aerial photographs could be used. \\ \hline
    F17 & Forb cover & \\ \hline
    F18 & Carex cover & \\ \hline
    F19 & Dominance of the most abundant herbaceous species & \\ \hline
    F20 & Cover of invasive plants & \\ \hline
    F21 & Invasive cover in the terrestrial environment & \\ \hline
    F22 & Fringing wetlands & \\ \hline
    F23 & Lacustrine wetlands & \\ \hline
    F24 & \% of the EA without surface water & \\ \hline
    F25 & \% of the EA with persistent surface water & \\ \hline
    F26 & \% of shaded water in summer & \\ \hline
    F27 & \% of the EA that is seasonally flooded only & \\ \hline
    F28 & Annual water fluctuation & \\ \hline
    F29 & Predominant depth class & \\ \hline
    F30 & Depth class – Uniformity of proportions & \\ \hline
    F31 & \% of stagnant water (non-flowing) & \\ \hline
    F32 & Minimum size – Stagnant open water & \\ \hline
    F33 & \% of accumulated water that is open & \\ \hline
    F34 & Width of the vegetation zone in the wetland & The mosaic of very high-resolution spatial aerial photographs could be used. \\ \hline
    F35 & Extent of flat shoreline & Mosaic of very high-resolution spatial aerial photographs and high-resolution slope layer. Classification of slope percentage (0-5\%; > 5\%). \\ \hline
    F36 & Robust emergent vegetation & \\ \hline
    F37 & Intersection of emergent vegetation with open waters & \\ \hline
    F38 & Extent of persistent deep water & \\ \hline
    F39 & Non-vegetated aquatic cover & \\ \hline
    F40 & Isolated island & \\ \hline
    F41 & Floating algae and duckweed & \\ \hline
    F42 & Connection to the canal and duration of outgoing flow & The “Wet Areas Mapping” layer could be used. Since we were able to conduct tests with the multiprobe during the driest period in most wetlands, the conclusion to this question was that the water was persistent for the majority of the wetlands, except for W18, where the stream was dry. \\ \hline
    F43 & Outflow confinement & The very high spatial resolution DEM could be used. \\ \hline
    F44 & Tributary channels & The “Wet Areas Mapping” layer could be used. \\ \hline
    F45 & Inflow water temperature & Use of the multiprobe. \\ \hline
    F46 & Flow resistance & \\ \hline
    F47 & pH measurement & Use of the multiprobe (use a comma instead of a dot to record the response). \\ \hline
    F48 & TDS and/or conductivity & Use of the multiprobe (use a comma instead of a dot to record the response). \\ \hline
    F49 & Beaver probability & \\ \hline
    F50 & Evidence of groundwater & High-resolution slope layers and “Wet Areas Mapping” layers could be used. \\ \hline
    F51 & Internal gradient & Use of precise contour lines, level of the inlet minus the level of the outlet divided by the distance traveled by the stream, expressed as a percentage (all in an Excel file). \\ \hline
    F52 & \% of the perimeter's vegetative buffer & The mosaic of very high-resolution spatial aerial photographs and the Forest layer (location of plantations) could be used, with a 30m buffer for precision. \\ \hline
    F53 & Buffer zone cover type & The mosaic of very high-resolution spatial aerial photographs could be used. \\ \hline
    F54 & Buffer zone slope & The mosaic of very high-resolution spatial aerial photographs and the high-resolution slope layer is used. The slope layer is divided into three classes, based on the type of slope. \\ \hline
    F55 & Steep cliffs or banks & 1 – Mosaic of very high-resolution spatial aerial photographs: Absence of vegetation? 2 – Very high spatial resolution DEM: classification with 2 m elevation classes (max. elevation = 515 m - min. elevation = 151 m). 3 – Validation with the topographic profile: 2 m elevation change? \\ \hline
    F56 & New or extended wetland & Road and forest path layers and the chronosequence of aerial photograph mosaics could be used. \\ \hline
    F57 & Fire history & National Burned Area Composite. \\ \hline
    F58 & Visibility & \\ \hline
    F59 & Non-consumptive uses - actual or potential & \\ \hline
    F60 & Central zone not visited & \\ \hline
    F61 & Frequently visited area & \\ \hline
    F62 & BMP - Soils & Always 0 in the context of this project. \\ \hline
    F63 & BMP - Wildlife protection & Always 0 in the context of this project. \\ \hline
    F64 & Consumptive uses (provisioning services) & \\ \hline
    F65 & Drinking water & Since the designated watershed provides drinking water, all wetlands have drinking water within 100 m of the EA. \\ \hline
    F66 & Calcareous minerotrophic peatland & Always 0 in the context of this project. The characteristics of this ecosystem should be better studied. Forest soil layers, bedrock geology, and “Wet Areas Mapping” layers should be considered. \\ \hline
    S1 & Abnormal inflow patterns & The mosaic of very high-resolution spatial aerial photographs could be used, and the chronosequence of aerial photograph mosaics could be used. \\ \hline
    S2 & Accelerated input of contaminants and/or salts into the wetland or its contribution area & \\ \hline
    S3 & Accelerated nutrient input into the wetland or its contribution area & \\ \hline
    S4 & Excessive sediment load from the contribution area & The mosaic of very high-resolution spatial aerial photographs could be used, and the chronosequence of aerial photograph mosaics could be used. \\ \hline
    S5 & Soil or sediment degradation within the evaluation area & The mosaic of very high-resolution spatial aerial photographs could be used, and the chronosequence of aerial photograph mosaics could be used. \\ \hline
\end{longtable}