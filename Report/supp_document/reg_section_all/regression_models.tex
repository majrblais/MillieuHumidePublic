\begin{enumerate}
    \item \textbf{Ridge Regression}: Adds a penalty equal to the square of the magnitude of coefficients to the least squares cost function to control the complexity of the model. Useful to handle multicollinearity and to prevent overfitting.

    \item \textbf{Decision Tree Regressor}: Uses a decision tree to go from observations about an item to conclusions about the item's target value. It is non-parametric and capable of capturing non-linear relationships.

    \item \textbf{Random Forest Regressor}: An ensemble of decision trees, typically trained via the bagging method. It is effective for regression tasks with high variance or with data that includes outliers.

    \item \textbf{Gradient Boosting Regressor}: Builds an additive model in a forward stage-wise fashion; it allows for the optimization of arbitrary differentiable loss functions, providing a robust approach to handle various types of regression problems.

    \item \textbf{AdaBoost Regressor}: An adaptive boosting algorithm that weights instances in the dataset by how easy or difficult they are to predict, allowing it to focus on the harder cases which increases the model's robustness.

    \item \textbf{K-Neighbors Regressor}: A non-parametric method wherein the prediction for a point is based on the mean or median of its nearest neighbors. It is highly adaptable and simple, but can become computationally expensive as data grows.

    \item \textbf{MLP Regressor (Multi-layer Perceptron)}: A type of neural network suitable for complex regression problems with large datasets, capable of capturing intricate patterns in data through its layers and neurons.

    \item \textbf{ElasticNet}: Combines penalties of both Lasso (L1) and Ridge (L2) regression methods in its cost function, which can be useful when there are multiple features correlated with each other.

    \item \textbf{SGD Regressor (Stochastic Gradient Descent)}: Optimizes the same cost function as ordinary linear regression using a stochastic gradient descent approach. This is useful for large-scale and sparse machine learning problems.

    \item \textbf{SVR (Support Vector Regression)}: Uses the same principles as SVM for classification, with support vectors and margin maximization, adjusted for regression. Effective for high-dimensional spaces.

    \item \textbf{Bayesian Ridge}: Incorporates Bayesian inference into the regression, providing a probabilistic approach which results in a distribution of possible outcomes, rather than single point estimates.

    \item \textbf{Kernel Ridge Regression}: Combines Ridge Regression with the kernel trick, making it suitable for non-linear problems. It is versatile in handling different types of data structures.

    \item \textbf{Linear Regression}: The simplest form of regression, fitting a linear equation to observed data. It provides a straightforward solution to the regression problem.

    \item \textbf{RANSAC Regressor}: Uses a robust method to fit a regression model to a subset of the data, the so-called inliers, and is particularly useful when data is contaminated with outliers.

    \item \textbf{TheilSen Regressor}: Robust multivariate regression model which is less sensitive to outliers. It can be more effective than ordinary least squares in situations with high levels of noise.
\end{enumerate}